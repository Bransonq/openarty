\documentclass{gqtekspec}
%%%%%%%%%%%%%%%%%%%%%%%%%%%%%%%%%%%%%%%%%%%%%%%%%%%%%%%%%%%%%%%%%%%%%%%%%%%%%%%%
%%
%% Filename: 	spec.tex
%%
%% Project:	OpenArty, an entirely open SoC based upon the Arty platform
%%
%% Purpose:	
%%
%% Creator:	Dan Gisselquist, Ph.D.
%%		Gisselquist Technology, LLC
%%
%%%%%%%%%%%%%%%%%%%%%%%%%%%%%%%%%%%%%%%%%%%%%%%%%%%%%%%%%%%%%%%%%%%%%%%%%%%%%%%%
%%
%% Copyright (C) 2015-2016, Gisselquist Technology, LLC
%%
%% This program is free software (firmware): you can redistribute it and/or
%% modify it under the terms of  the GNU General Public License as published
%% by the Free Software Foundation, either version 3 of the License, or (at
%% your option) any later version.
%%
%% This program is distributed in the hope that it will be useful, but WITHOUT
%% ANY WARRANTY; without even the implied warranty of MERCHANTIBILITY or
%% FITNESS FOR A PARTICULAR PURPOSE.  See the GNU General Public License
%% for more details.
%%
%% You should have received a copy of the GNU General Public License along
%% with this program.  (It's in the $(ROOT)/doc directory, run make with no
%% target there if the PDF file isn't present.)  If not, see
%% <http://www.gnu.org/licenses/> for a copy.
%%
%% License:	GPL, v3, as defined and found on www.gnu.org,
%%		http://www.gnu.org/licenses/gpl.html
%%
%%
%%%%%%%%%%%%%%%%%%%%%%%%%%%%%%%%%%%%%%%%%%%%%%%%%%%%%%%%%%%%%%%%%%%%%%%%%%%%%%%%
%%
%%
\usepackage{import}
\usepackage{bytefield}
\project{OpenArty}
\title{Specification}
\author{Dan Gisselquist, Ph.D.}
\email{dgisselq (at) opencores.org}
\revision{Rev.~0.0}
\begin{document}
\pagestyle{gqtekspecplain}
\titlepage
\begin{license}
Copyright (C) \theyear\today, Gisselquist Technology, LLC

This project is free software (firmware): you can redistribute it and/or
modify it under the terms of  the GNU General Public License as published
by the Free Software Foundation, either version 3 of the License, or (at
your option) any later version.

This program is distributed in the hope that it will be useful, but WITHOUT
ANY WARRANTY; without even the implied warranty of MERCHANTIBILITY or
FITNESS FOR A PARTICULAR PURPOSE.  See the GNU General Public License
for more details.

You should have received a copy of the GNU General Public License along
with this program.  If not, see \texttt{http://www.gnu.org/licenses/} for a copy.
\end{license}
\begin{revisionhistory}
0.0 & 6/20/2016 & Gisselquist & First Draft \\\hline
\end{revisionhistory}
% Revision History
% Table of Contents, named Contents
\tableofcontents
\listoffigures
\listoftables
\begin{preface}
\end{preface}

\chapter{Introduction}
\pagenumbering{arabic}
\setcounter{page}{1}

% What is old
%	Arty, XuLA
% What does the old lack?
%	Arty lacks open interfaces, instead using MIG and CoreGen w/ AXI bus
% What is new
%	OpenArty has its own memory interface controller, and runs everything
%	off of an open Wishbone bus structure.
% What does the new have that the old lacks
%
% What performance gain can be expected?
%

The goals of this project include:
\begin{enumerate}
\item Use entirely open interfaces

	This means not using the Memory Interface Generator (MIG), the
	Xilinx CoreGen IP, etc.  Further, I wish to use all of Arty's on--board
	hardware: Flash, DDR3-SDRAM, Ethernet, and everything else at their
	full and fastest speed(s).  For example, the flash will need to be
	clocked at 82~MHz, not the 50~MHz I've clocked it at before.  The
	memory should also be able to support pipelined 32--bit interactions
	over the Wishbone bus at a 162~MHz clock.  Finally, the Ethernet
	controller should be supported by a DMA capable interface that can
	drive the ethernet at its full 100Mbps rate.

\item Run using a 162.5~MHz clock, if for no other reason than to gain the
	experience of building logic that can run that fast.\footnote{The
	original goal was to run at 200~MHz.  However, the memory controller
	cannot run faster than 83~MHz.  If we run it at 81.25~MHz and double
	that clock to get our logic clock, that now places us at 162.5~MHz.
	200~MHz is \ldots too fast for DDR3 transfers using the Artix--7 chip
	on the Arty.}

\item Modify the ZipCPU to support an MMU and a data cache, and perhaps even
	a floating point unit.

\item The default configuration will also include three Pmods: a USBUART,
	an SDCard, and the GPS Pmod.
\end{enumerate}

I intend to demonstrate this project with a couple programs:
\begin{enumerate}
\item A very simple program that runs automatically upon startup that can be
	used to select from among multiple configurations.
\item NTP Server
\item A ZipOS that can actually load and run programs from the SD Card
\end{enumerate}

\chapter{Architecture}


\chapter{Software}
\section{Directory Structure}
\section{Zip CPU Tool Chain}
\section{Bench Test Software}
\section{Host Software}
\begin{itemize}
\item {\tt readflash}: As I am loathe to remove anything from
	a device that came factory installed, the
	{\tt readflash} program reads the original installed
	configuration from the flash and dumps it to a file.

\item {\tt wbregs}: This program offers a capability very similar to the 
	PEEK and POKE capability Apple user's may remember from before the
	days of Macintosh.  {\tt wbregs <address>} will read from the
	Wishbone bus the value at the given address.  Likewise
	{\tt wbregs <address> <value>} will write the given value into the
	given address.  While both address and value have the semantics of
	numbers acceptable to {\tt strtoul()}, the address can also be a named
	address.  Supported names can be found in {\tt regdefs.cpp}, and their
	register mapping in {\tt regdefs.h}.
\item {\tt ziprun}:
\item {\tt zipload}:
\end{itemize}

\section{Zip CPU Programs}
\begin{itemize}
\item {\tt ntpserver}:
\item {\tt goldenstart}:
\end{itemize}
\section{ZipOS}
\subsection{System Calls}
\begin{itemize}
\item {\tt int wait(unsigned event\_mask, int timeout)}
\item {\tt int clear(unsigned event\_mask, int timeout)}
\item {\tt void post(unsigned event\_mask)}
\item {\tt void yield(void) }
\item {\tt int read(int fid, void *buf, int len)}
\item {\tt int write(int fid, void *buf, int len)}
\item {\tt unsigned time(void) }
% \item SEMGET
% \item SEMPUT
\item {\tt void *malloc(void)}
\item {\tt void free(void *buf)}
% \item FORK
% \item opendir
% \item EXEC
% \item OPEN
\end{itemize}
\subsection{Scheduler}

\chapter{Operation}

\chapter{Registers}
There are several address regions on the S6~SoC, as shown in
Tbl.~\ref{tbl:memregions}.
\begin{table}[htbp]
\begin{center}\begin{tabular}{|p{2.25in}|p{0.6in}|p{0.45in}|p{2.0in}|}\hline
\rowcolor[gray]{0.85} Binary Address & Base & Size(W) & Purpose \\\hline\hline
\scalebox{0.9}{\tt 0000 0000 0000 0000 0001 000x xxxx} & \scalebox{0.9}{\tt 0x00000100} & \hfill 32 & Peripheral I/O Control \\\hline
\scalebox{0.9}{\tt 0000 0000 0000 0000 0001 0010 0yyx} & \scalebox{0.9}{\tt 0x00000120} & \hfill 8 & Debug scope control\\\hline
\scalebox{0.9}{\tt 0000 0000 0000 0000 0001 0010 10xx} & \scalebox{0.9}{\tt 0x00000128} & \hfill 4 & RTC control\\\hline
\scalebox{0.9}{\tt 0000 0000 0000 0000 0001 0010 11xx} & \scalebox{0.9}{\tt 0x0000012c} & \hfill 4 & SDCard controller\\\hline
\scalebox{0.9}{\tt 0000 0000 0000 0000 0001 0011 00xx} & \scalebox{0.9}{\tt 0x00000130} & \hfill 4 & GPS Clock loop control\\\hline
\scalebox{0.9}{\tt 0000 0000 0000 0000 0001 0011 01xx} & \scalebox{0.9}{\tt 0x00000134} & \hfill 4 & OLEDrgb control\\\hline
\scalebox{0.9}{\tt 0000 0000 0000 0000 0001 0011 1xxx} & \scalebox{0.9}{\tt 0x00000138} & \hfill 8 & Network packet interface\\\hline
\scalebox{0.9}{\tt 0000 0000 0000 0000 0001 0100 0xxx} & \scalebox{0.9}{\tt 0x00000140} & \hfill 8 & GPS Testbench\\\hline
\scalebox{0.9}{\tt 0000 0000 0000 0000 0001 0100 1xxx} & \scalebox{0.9}{\tt 0x00000148} & \hfill  8 & {\em Unused}\\\hline
\scalebox{0.9}{\tt 0000 0000 0000 0000 0001 0101 xxxx} & \scalebox{0.9}{\tt 0x00000150} & \hfill 16 & {\em Unused}\\\hline
\scalebox{0.9}{\tt 0000 0000 0000 0000 0001 011x xxxx} & \scalebox{0.9}{\tt 0x00000160} & \hfill 32 & {\em Unused}\\\hline
\scalebox{0.9}{\tt 0000 0000 0000 0000 0001 100x xxxx} & \scalebox{0.9}{\tt 0x00000180} & \hfill 32 & {\em Unused}\\\hline
\scalebox{0.9}{\tt 0000 0000 0000 0000 0001 101x xxxx} & \scalebox{0.9}{\tt 0x000001a0} & \hfill 32 & Ethernet configuration registers\\\hline
\scalebox{0.9}{\tt 0000 0000 0000 0000 0001 110x xxxx} & \scalebox{0.9}{\tt 0x000001c0} & \hfill 32 & Extended Flash Control Port\\\hline
\scalebox{0.9}{\tt 0000 0000 0000 0000 0001 111x xxxx} & \scalebox{0.9}{\tt 0x000001e0} & \hfill 32 & ICAPE2 Configuration Port\\\hline
\scalebox{0.9}{\tt 0000 0000 0000 0000 10xx xxxx xxxx} & \scalebox{0.9}{\tt 0x00000800} & \hfill 1k & Ethernet RX Buffer\\\hline
\scalebox{0.9}{\tt 0000 0000 0000 0000 11xx xxxx xxxx} & \scalebox{0.9}{\tt 0x00000c00} & \hfill 1k & Ethernet TX Buffer\\\hline
\scalebox{0.9}{\tt 0000 0000 0000 1xxx xxxx xxxx xxxx} & \scalebox{0.9}{\tt 0x00008000} & \hfill 32k & On-chip Block RAM\\\hline
\scalebox{0.9}{\tt 0000 01xx xxxx xxxx xxxx xxxx xxxx} & \scalebox{0.9}{\tt 0x00400000} & \hfill 4M & QuadSPI Flash\\\hline
\scalebox{0.9}{\tt 0100 0000 0000 0000 0000 0000 0000} & \scalebox{0.9}{\tt 0x0400000} & & Configuration Start\\\hline
\scalebox{0.9}{\tt 0100 1000 0000 0000 0000 0000 0000} & \scalebox{0.9}{\tt 0x0480000} & & CPU Reset Address\\\hline
\scalebox{0.9}{\tt 01xx xxxx xxxx xxxx xxxx xxxx xxxx} & \scalebox{0.9}{\tt 0x04000000} & \hfill 64M & DDR3 SDRAM\\\hline
\scalebox{0.9}{\tt 1000 0000 0000 0000 0000 0000 000x} & \scalebox{0.9}{\tt 0x08000000} & \hfill 2 & ZipCPU debug control port---only visible to debug WB master\\\hline
\end{tabular}
\caption{Address Regions}\label{tbl:memregions}
\end{center}\end{table}

\begin{table}[htbp]
\begin{center}\begin{tabular}{|p{0.9in}|p{0.45in}|p{3.5in}|}\hline
\rowcolor[gray]{0.85} Base & Size(W) & Purpose \\\hline\hline
\scalebox{0.9}{\tt 0x0c0000000} & 1 & Primary Zip PIC\\\hline
\scalebox{0.9}{\tt 0x0c0000001} & 1 & Watchdog Timer\\\hline
\scalebox{0.9}{\tt 0x0c0000002} & 1 & Bus Watchdog Timer\\\hline
\scalebox{0.9}{\tt 0x0c0000003} & 1 & Alternate Zip PIC\\\hline
\scalebox{0.9}{\tt 0x0c0000004} & 1 & ZipTimer-A\\\hline
\scalebox{0.9}{\tt 0x0c0000005} & 1 & ZipTimer-B\\\hline
\scalebox{0.9}{\tt 0x0c0000006} & 1 & ZipTimer-C\\\hline
\scalebox{0.9}{\tt 0x0c0000007} & 1 & ZipJiffies\\\hline
\scalebox{0.9}{\tt 0x0c0000008} & 1 & Master task counter\\\hline
\scalebox{0.9}{\tt 0x0c0000009} & 1 & Master prefetch stall counter\\\hline
\scalebox{0.9}{\tt 0x0c000000a} & 1 & Master memory stall counter\\\hline
\scalebox{0.9}{\tt 0x0c000000b} & 1 & Master instruction counter\\\hline
\scalebox{0.9}{\tt 0x0c000000c} & 1 & User task counter\\\hline
\scalebox{0.9}{\tt 0x0c000000d} & 1 & User prefetch stall counter\\\hline
\scalebox{0.9}{\tt 0x0c000000e} & 1 & User memory stall counter\\\hline
\scalebox{0.9}{\tt 0x0c000000f} & 1 & User instruction counter\\\hline
\scalebox{0.9}{\tt 0x0c0000010} & 1 & DMA command register\\\hline
\scalebox{0.9}{\tt 0x0c0000011} & 1 & DMA length\\\hline
\scalebox{0.9}{\tt 0x0c0000012} & 1 & DMA source address\\\hline
\scalebox{0.9}{\tt 0x0c0000013} & 1 & DMA destination address\\\hline
\scalebox{0.9}{\tt 0x0c0000040} & 1 & {\em Reserved for MMU context register}\\\hline
\scalebox{0.9}{\tt 0x0c0000080} & 32 & {\em Reserved for MMU TLB}\\\hline
\end{tabular}
\caption{ZipSystem Addresses}\label{tbl:zipio}
\end{center}\end{table}

\section{Peripheral I/O Control}
Tbl.~\ref{tbl:ioregs}
\begin{table}[htbp]
\begin{center}\begin{reglist}
VERSION  &\scalebox{0.8}{\tt 0x0100} & 32 & R & Build date\\\hline
PIC      &\scalebox{0.8}{\tt 0x0101} & 32 & R/W & Bus Interrupt Controller \\\hline
BUSERR   &\scalebox{0.8}{\tt 0x0102} & 32 & R & Last Bus Error Address\\\hline
PWRCOUNT &\scalebox{0.8}{\tt 0x0103} & 32 & R & Ticks since startup\\\hline
BTNSW    &\scalebox{0.8}{\tt 0x0104} & 32 & R/W & Button/Switch controller\\\hline
LEDCTRL  &\scalebox{0.8}{\tt 0x0105} & 32 & R/W & LED Controller \\\hline
AUXSETUP &\scalebox{0.8}{\tt 0x0106} & 29 & R/W & Auxilliary UART config\\\hline
GPSSETUP &\scalebox{0.8}{\tt 0x0107} & 29 & R/W & GPS UART config\\\hline
CLR-LEDx &\scalebox{0.8}{\tt 0x0108-b} & 32 & R/W & Color LED controller\\\hline
RTCDATE  &\scalebox{0.8}{\tt 0x010c} & 32 & R/W & BCD Calendar Date\\\hline
GPIO     &\scalebox{0.8}{\tt 0x010d} & 32 & R/W & GPIO controller\\\hline
UARTRX   &\scalebox{0.8}{\tt 0x010e} & 32 & R/W & Aux UART receive byte\\\hline
UARTTX   &\scalebox{0.8}{\tt 0x010f} & 32 & R/W & Aux UART transmit byte\\\hline
GPSRX    &\scalebox{0.8}{\tt 0x0110} & 32 & R/W & GPS UART receive byte\\\hline
GPSTX    &\scalebox{0.8}{\tt 0x0111} & 32 & R/W & GPS UART transmit byte\\\hline
% 0x010c-0x010f
\end{reglist}
\caption{I/O Peripheral Registers}\label{tbl:ioregs}
\end{center}\end{table}
shows the addresses of various I/O peripherals included as part of the SoC.
We'll walk through each of these peripherals in turn, describing how they work.

\subsection{Interrupt Controller}
The OpenArty design maintains three interrupt controllers.  Two of them
are found within the ZipSystem, and the third is located on the bus
itself.  Of these, the primary interrupt controller is located in the ZipSystem.
This interrupt controller accepts, as interrupt inputs, the outputs of both
the auxilliary interrupt controller as well as the bus interrupt controller.
Hence, even though the CPU only supports a single interrupt line, by using
these three interrupt controllers many more interrupts can be supported.

The primary interrupt controller handles interrupts from the sources listed
in Tbl.~\ref{tbl:sys-ints}.  These interrupts are listed together with the
mask that would need to be used when referencing them to the interrupt
controller.  In a similar fashion, the auxilliary interrupt controller accepts
inputs from the sources listed in Tbl.~\ref{tbl:aux-ints}.  Finally, the
bus interrupt controller handles the interrupts from the sources listed in
Tbl.~\ref{tbl:bus-ints}.

\begin{table}[htbp]
\begin{center}\begin{tabular}{|p{0.9in}|p{0.75in}|p{0.75in}|p{3.00in}|}\hline
\rowcolor[gray]{0.85} Name & Bit Mask & DMAC ID &Description \\\hline\hline
SYS\_DMAC  & 0x0001 && The DMA controller is idle.\\\hline
SYS\_JIF   & 0x0002 & 1 & A Jiffies timer has expired.\\\hline
SYS\_TMC   & 0x0004 & 2 & Timer C has timed out.\\\hline
SYS\_TMB   & 0x0008 & 3 & Timer C has timed out.\\\hline
SYS\_TMA   & 0x0010 & 4 & Timer C has timed out.\\\hline
SYS\_AUX   & 0x0020 & 5 & The auxilliary interrupt controller sends an interrupt\\\hline
SYS\_EXT   & 0x0040 & 6 & A Bus interrupt has tripped. \\\hline
SYS\_PPS   & 0x0080 & 7 & An interrupt marking the top of the second\\\hline
SYS\_GPSRX & 0x0100 & 8& A character has been received via GPS\\\hline
SYS\_NETRX & 0x0200 & 9 & A packet has been received via the network\\\hline
SYS\_NETTX & 0x0400 & 10 & The network controller is idle, having sent its
			last packet\\\hline
SYS\_UARTRX & 0x0800 & 11 & A character has been received via the UART\\\hline
SYS\_UARTTX & 0x1000 & 12 & The transmit UART is idle, and ready for its next
		character.\\\hline
SYS\_SDCARD & 0x2000 & 13 & The SD-Card controller has become idle\\\hline
SYS\_BUTTON & 0x4000 & 14 & A Button has been pressed. \\\hline
\end{tabular}
\caption{Primary System Interrupts}\label{tbl:sys-ints}
\end{center}\end{table}
%%%%%%%%%%%%%
\begin{table}[htbp]
\begin{center}\begin{tabular}{|p{0.9in}|p{0.75in}|p{0.75in}|p{3.00in}|}\hline
\rowcolor[gray]{0.85} Name & Bit Mask & DMAC ID &Description \\\hline\hline
AUX\_UIC & 0x0001 & 16 & The user instruction counter has overflowed.\\\hline
AUX\_UPC & 0x0002 & 17 & The user prefetch stall counter has overflowed.\\\hline
AUX\_UOC & 0x0004 & 18 & The user ops stall counter has overflowed.\\\hline
AUX\_UTC & 0x0008 & 19 & The user clock tick counter has overflowed.\\\hline
AUX\_MIC & 0x0010 & 20 & The supervisor instruction counter has overflowed.\\\hline
AUX\_MPC & 0x0020 & 21 & The supervisor prefetch stall counter has overflowed.\\\hline
AUX\_MOC & 0x0040 & 22 & The supervisor ops stall counter has overflowed.\\\hline
AUX\_MTC & 0x0080 & 23 & The supervisor clock tick counter has overflowed.\\\hline
AUX\_SWITCH & 0x0100 & 24 & A switch has changed state\\\hline
AUX\_FLASH  & 0x0200 & 25 & The flash controller has completed a write/erase cycle\\\hline
AUX\_SCOPE & 0x0400 & 26 & The Scope has completed its collection\\\hline
AUX\_RTC    & 0x0800 & 27& An alarm or timer has taken place (assuming the RTC
		is installed, and includes both alarm or timer)\\\hline
AUX\_GPIO   & 0x1000 & 28 & The GPIO input lines have changed values.\\\hline
AUX\_OLED   & 0x2000 & 29 & The OLED driver is idle\\\hline
\end{tabular}
\caption{Auxilliary System Interrupts}\label{tbl:aux-ints}
\end{center}\end{table}

\begin{table}[htbp]
\begin{center}\begin{tabular}{|p{0.9in}|p{0.75in}|p{3.75in}|}\hline
\rowcolor[gray]{0.85} Name & Bit Mask & Description \\\hline\hline
BUS\_BUTTON & 0x0001 & A Button has been pressed. \\\hline
BUS\_SWITCH & 0x0002 & The Scope has completed its collection\\\hline
BUS\_PPS    & 0x0004 & Top of the second\\\hline
BUS\_RTC    & 0x0008 & An alarm or timer has taken place (assuming the RTC
		is installed, and includes both alarm or timer)\\\hline
BUS\_NETRX & 0x0010 & A packet has been received via the network\\\hline
BUS\_NETTX & 0x0020 & The network controller is idle, having sent its
			last packet\\\hline
BUS\_UARTRX & 0x0040 & A character has been received via the UART\\\hline
BUS\_UARTTX & 0x0080 & The transmit UART is idle, and ready for its next
		character.\\\hline
BUS\_GPIO   & 0x0100 & The GPIO input lines have changed values.\\\hline
BUS\_FLASH  & 0x0200 & The flash device has finished either its erase or
		write cycle, and is ready for its next command. (Alternate
	config only.)\\\hline
BUS\_SCOPE  & 0x0400 & A scope has completed collecting.\\\hline
BUS\_GPSRX  & 0x0800 & A character has been received via GPS\\\hline
BUS\_SDCARD & 0x1000 & The SD-Card controller has become idle\\\hline
BUS\_OLED   & 0x2000 & The OLED interface has become idle\\\hline
BUS\_ZIP    & 0x4000 & True if the ZipCPU has come to a halt\\\hline
\end{tabular}
\caption{Bus Interrupts}\label{tbl:bus-ints}
\end{center}\end{table}

\subsection{Last Bus Error Address}
\subsection{General Purpose I/O}
\subsection{UART Data Register}
\section{Debugging Scopes}
\section{Internal Configuration Access Port}
\section{Real--Time Clock}
\section{On-Chip Block RAM}
\section{Flash Memory}
\begin{table}
\begin{center}\begin{reglist}
ewreg  &\scalebox{0.8}{\tt 0x0180} & 32 & R & Erase/write control and status\\\hline
status      &\scalebox{0.8}{\tt 0x0181} & 8 & R/W & Bus Interrupt Controller \\\hline
nvconf   &\scalebox{0.8}{\tt 0x0182} & 16 & R & Last Bus Error Address\\\hline
vconf &\scalebox{0.8}{\tt 0x0183} & 8 & R & Ticks since startup\\\hline
evonc    &\scalebox{0.8}{\tt 0x0184} & 8 & R/W & Button/Switch controller\\\hline
lock  &\scalebox{0.8}{\tt 0x0185} & 8 & R/W & LED Controller \\\hline
flagstatus&\scalebox{0.8}{\tt 0x0186} & 8 & R/W & Auxilliary UART config\\\hline
clear	&\scalebox{0.8}{\tt 0x0187} & 8 & R/W & Clear status on write\\\hline
Device ID &\scalebox{0.8}{\tt 0x0188-}\hfill & 5x32 & R & Device ID\\
	&\scalebox{0.8}{\tt -0x018c}\hfill & & & \\\hline
% asyncID &\scalebox{0.8}{\tt 0x018d} & 32 & R/W & Asynch Read ID.  Write starts the ASynch read, 0xff returned until complete\\\hline
asyncOTP  &\scalebox{0.8}{\tt 0x18e} & 32 & W & Asynch Read OTP.  Write starts the ASynch read, 0xff returned until complete\\\hline
OTP     &\scalebox{0.8}{\tt 0x0190-}\hfill &16x32 & R/W & OTP Memory\\
	&\scalebox{0.8}{\hfill\tt -0x19f} & & & \\\hline
% 0x010c-0x010f
\end{reglist}
\caption{Flash control registers}\label{tbl:flctl}
\end{center}\end{table}

\chapter{Wishbone Datasheet}\label{ch:wishbone}

The master and slave interfaces have been simplified with the following
requirement: the {\tt STB} line is not allowed to be high unless the {\tt CYC}
line is high.  In this fashion, a slave may often be able to ignore {\tt CYC}
and only act on the presence of {\tt STB}, knowing that {\tt CYC} must be
active at the same time.

\chapter{Clocks}\label{ch:clocks}
\begin{table}\begin{center}
\begin{clocklist}
{\tt i\_clk\_100mhz} & Ext & \multicolumn{2}{c|}{100} &
	100~MHz Crystal Oscillator \\\hline
{\em Future }{\tt s\_clk} & PLL & 152 & 166 & Internal Logic, Wishbone Clock \\\hline
{\tt s\_clk} & PLL & 83.33 & 75.76& DDR3 SDRAM Controller Clock \\\hline
\multicolumn{2}{|c|}{\tt mem\_clk\_200mhz} & 200~MHz & & MIG Reference clock for PHASERs\\\hline
{\tt ddr3\_ck\_}$x$ & DDR & 166.67 & 303 & DDR3 Command Clock\\\hline
{\tt o\_qspi\_sck} & DDR & 95 & & QSPI Flash clock \\\hline
{\tt o\_sd\_clk} & Logic & 50 & 0.100 & SD--Card clock \\\hline
{\tt o\_oled\_sck} & Logic & 166 & & OLED SPI clock \\\hline
{\tt o\_eth\_mdclk} & Logic & 25 & 2.5 & Ethernet MDIO controller clock\\\hline
\end{clocklist}
\caption{OpenArty clocks}\label{tbl:clocks}
\end{center}\end{table}

\chapter{I/O Ports}

Table.~\ref{tbl:ioports}
\begin{table}[htbp]
\begin{center}
\begin{portlist}
i\_clk\_100mhz & 1 & Input & Clock\\\hline
o\_qspi\_cs\_n & 1 & Output & Quad SPI Flash chip select\\\hline
o\_qspi\_sck & 1 & Output & Quad SPI Flash clock\\\hline
io\_qspi\_dat & 4 & Input/Output & Four-wire SPI flash data bus\\\hline
i\_btn & 4 & Input  & Inputs from the two on-board push-buttons\\\hline
i\_sw  & 4 & Input  & Inputs from the two on-board push-buttons\\\hline
o\_led & 4 & Output & Outputs controlling the four on-board LED's\\\hline
o\_clr\_led0 & 3 & Output & \\\hline
o\_clr\_led1 & 3 & Output & \\\hline
o\_clr\_led2 & 3 & Output & \\\hline
o\_clr\_led3 & 3 & Output & \\\hline
i\_uart\_rx & 1 & Input &  UART receive input\\\hline
o\_uart\_tx & 1 & Output & UART transmit output\\\hline\hline
i\_aux\_rx & 1 & Input &  Auxiliary/Pmod UART receive input\\\hline
o\_aux\_tx & 1 & Output & Auxiliary/Pmod UART transmit output\\\hline
i\_aux\_rts & 1 & Input &  Auxiliary/Pmod UART receive input\\\hline
o\_aux\_cts & 1 & Output & Auxiliary/Pmod UART transmit output\\\hline\hline
i\_gps\_rx & 1 & Input &  GPS/Pmod UART receive input\\\hline
o\_gps\_tx & 1 & Output & GPS/Pmod UART transmit output\\\hline
i\_gps\_pps & 1 & Input & GPS Part-per-second (PPS) signal\\\hline
i\_gps\_3df & 1 & Input & GPS\\\hline\hline
o\_oled\_cs\_n & 1 & Output & \\\hline
o\_oled\_sck & 1 & Output & \\\hline
o\_oled\_mosi & 1 & Output & \\\hline
i\_oled\_miso & 1 & Input & \\\hline
o\_oled\_reset & 1 & Output & \\\hline
o\_oled\_dc & 1 & Output & \\\hline
o\_oled\_en & 1 & Output & \\\hline
o\_oled\_pmen & 1 & Output & \\\hline\hline
o\_sd\_sck & 1 & Output & SD Clock\\\hline
i\_sd\_cd & 1 & Input & Card Detect\\\hline
i\_sd\_wp & 1 & Input & Write Protect\\\hline
io\_cmd & 1 & In/Output & SD Bi-directional command wire\\\hline
io\_sd & 4 & In/Output & SD Bi-directional data lines\\\hline\hline
o\_cls\_cs\_n & 1 & Output & CLS Display chip select\\\hline
o\_cls\_sck & 1 & Output & CLS Display clock\\\hline
o\_cls\_mosi & 1 & Output & CLS Display MOSI\\\hline
i\_cls\_miso & 1 & Input & CLS Display MISO\\\hline\hline
\end{portlist}
\caption{List of IO ports}\label{tbl:ioports}
\end{center}\end{table}
lists the various I/O ports associated with OpenArty.


% Appendices
% Index
\end{document}


